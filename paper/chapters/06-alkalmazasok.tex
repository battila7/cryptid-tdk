\chapter{Alkalmazások}
\label{Chapter::Alkalmazasok}

Az \dotref{Section::CryptID::ApplicationDevelopment} alfejezet API-használatra fókuszáló példaprogramja után ebben a fejezetben nagyobb léptékű alkalmazásokat mutatunk be. A fejezet célja, hogy valós, gyakorlati példákat adjon olyan webalkalmazásokra, melyek kulcsösszetevője lehet az IBE, s ezáltal a CryptID.

Előbb egy ténylegesen implementált webalkalmazást – CrpytID.email – ismertetünk, majd Személyre szabott zárthelyi néven egy rövid esettanulmány formájában vázoljuk az IBE egy lehetséges egyetemi felhasználását.

\section{Implementáció – CryptID.email}

A CryptID.email egy fájltitkosításra szolgáló platformfüggetlen webalkalmazás, melynek legfontosabb jellemzői a következők:

\begin{outdentlist}
    \item[]\textbf{Email, mint azonosító.}
    Az IBE-t használó alkalmazások implementációjában kulcsfontosságú kérdés, hogy mit használjunk azonosítóként. Míg egy vállalati belső alkalmazás esetén ez a legtöbbször adott (például LDAP útján), addig egy nyílt, tetszőleges felhasználó által igénybe vehető szolgáltatás esetén ez korántsem egyértelmű.
    
    Az azonosító kiválasztásának két kiemelkedő szempontja az azonosító elterjedtsége (hány felhasználó rendelkezik vele?) valamint ellenőrizhetősége (mennyire bonyolult ellenőrizni, hogy egy azonosító valóban egy adott entitáshoz tartozik?). Ezek figyelembevételével a CrpytID.email esetén az email címet választottuk azonosítóként, hiszen az internetfelhasználók zöme rendelkezik legalább egy email címmel, valamint ellenőrizni is egyszerűbb, mint egy telefonszámot vagy egy Facebook-hozzáférést.

    \item[]\textbf{Szimmetrikus tartalomtitkosítás.}
    Az aszimmetrikus módszerek (mint az IBE) alacsony teljesítményük miatt nem alkalmasak nagyobb adattömeg titkosítására. Emiatt gyakori megoldás, hogy az aszimmetrikus kriptográfiát egy nagyságrendekkel gyorsabb szimmetrikus módszer kulcsának titkosítására használják. 

    Ezen ötlet jelenik meg a CryptID.email implementációjában is: Először generálunk egy véletlenszerű bájtsorozatot, mely a fájlok tartalmának titkosításához használt AES (\textit{Advanced Encryption Standard}) eljárás kulcsa lesz. Az AES titkosítás végeztével pedig az említett kulcsot a CryptID könyvtárral aszimmmetrikusan titkosítjuk.

    Ennek köszönhetően egyszerre élvezhetjük az IBE nyújtotta előnyöket, valamint a szimmetrikus titkosítás gyorsaságát.

    \item[]\textbf{Kliensoldali titkosítás.}
    Az IBE meghatározó jellemzője, hogy a visszafejtésre alkalmas privát kulcsokat egy megbízható fél, a PKG generálja. Ez azonban azt is jelenti, hogy a PKG tetszőleges publikus kulcshoz képes privát kulcsot előállítani. Ugyanakkor ez a lehetőség csak akkor jelent kockázatot, ha a PKG (vagy valamely, azzal közvetlen kapcsolatban álló komponens) a publikus kulcson kívül az azzal titkosított adatokhoz is hozzáfér.

    Ennek elkerülésére a CryptID.email kliensoldali titkosítást valósít meg, azaz mind a titkosítás, mind a visszafejtés úgy történik, hogy a publikus kulcson kívül semmilyen adat nem hagyja el a felhasználó eszközét.

\end{outdentlist}

A következőkben a titkosítás, majd a visszafejtés lépésein keresztül ismertetjük az alkalmazás megvalósításának részleteit.

\subsection{Titkosítás}

\subsubsection{A fájl beolvasása}

A titkosítási folyamat első lépése – természetesen a megfelelő webhely felkeresését követően – a titkosítandó fájl beolvasása. Ez a fájl teljes tartalmának memóriába történő beolvasását jelenti. Ugyan a CryptID.email által használt FileReader API\footnote{\url{https://w3c.github.io/FileAPI/#dfn-filereader} – \url{https://caniuse.com/#feat=filereader}} lehetővé teszi, hogy egyszerre csak a fájl egy kisebb szeletét olvassuk be és dolgozzuk fel (azaz \textit{streameljük}), azonban a böngészők jelenleg nem biztosítanak hatékony \textit{streaming} titkosítási eljárásokat, így ennek megfelelő kihasználása nem lehetséges.

\subsubsection{A fájl tartalmának titkosítása}

Miután a memóriában rendelkezésre áll a titkosítandó adat, a CryptID.email a WebCrypto API-n\footnote{\url{https://www.w3.org/TR/WebCryptoAPI/} – \url{https://caniuse.com/#feat=cryptography}} keresztül előbb generál egy AES kulcsot, majd ugyanezen API-t használva AES-GCM (\textit{Galois/Counter Mode}) blokktitkosítással rejtjelezi a bemenetet.

Ezzel a memóriában már készen áll egy szimmetrikus privát kulcs, valamint az ezzel titkosított adattömeg.

\subsubsection{Címzett kiválasztása}

Az IBE titkosítás megkezdése előtt a felhasználónak meg kell adnia a címzett email címét, ami a publikus kulcsban szereplő azonosítót jelenti.

\subsubsection{Publikus paraméterek lekérése}

A szimmetrikus kulcs titkosítása előtt szükséges a publikus IBE paraméterek beszerzése. Ezeket a szerver szolgáltatja, mely a paramétereken felül egy azokhoz köthető azonosítót is elküld a kliensnek. Erre azért van szükség, mert a szerver havonta új publikus paramétereket és mesterkulcsot generál (a \texttt{CryptID.setup} függvényt igénybe véve), viszont a korábban titkosított adatok feloldására ezek természetesen nem alkalmasak, így tárolni és azonosítani kell őket.

\subsubsection{A szimmetrikus kulcs titkosítása}

A következő lépés a fájl tartalmának rejtjelezéséhez használt szimmetrikus kulcs titkosítása. Ez a CryptID könyvtár \texttt{encrypt} függvényén keresztül történik, az előző lépés során megszerzett publikus paraméterek és az email címből képzett publikus kulcs segítségével.

\subsubsection{A titkos fájl összeállítása}

A CryptID.email a titkosított tartalmat nem juttatja el a címzetthez – arról a felhasználónak kell gondoskodnia. Ennek elősegítésére a CryptID.email létrehoz egy titkos fájlt, mely a következő összetevőket foglalja magában:

\begin{outdentlist}
    \item[]\textbf{Eredeti fájlnév.}
    A titkosított fájlhoz tartozó eredeti fájlnévre elsősorban a kiterjesztés megőrzése miatt van szükség, mely számos operációs rendszeren kitüntetett jelentéssel bír.

    \item[]\textbf{Publikus paraméterek.}
    A szimmetrikus kulcs titkosításához felhasznált publikus IBE paraméterek azonosítója. A CryptID.email szerveroldali komponense ez alapján tudja megkeresni a privát kulcs generálásához szükség mesterkulcsot.

    \item[]\textbf{Titkosított szimmetrikus kulcs.}
    Az IBE-titkosított AES kulcs.

    \item[]\textbf{Titkosított fájltartalom.} 
    Az eredeti fájl tartalma AES-titkosítva.
\end{outdentlist}

A titkos fájlt ezután a felhasználó a memóriából a háttértárra mentheti, majd igény szerint továbbíthatja.

\subsection{Visszafejtés}

A visszafejtés lépéssora ott veszi fel a fonalat, hogy a felhasználó valamilyen módon hozzájutott egy, a CryptID.emailen keresztül előállított titkos fájlhoz, és szeretné visszafejteni az ebben tartalmazott eredeti fájlt. A visszafejtés azonban csak egy megfelelő privát kulcs birtokában sikerülhet. A privát kulcs beszerzésehez pedig a felhasználónak egy azonosítási folyamaton keresztül igazolnia kell az email címét.

Tekintve, hogy a visszafejtési folyamat a titkosítás inverze, megjelennek hasonló lépések, mint például a fájl memóriába olvasása. Ezek részletes ismertetését itt elhagyjuk, és csak a lényeges pontokat mutatjuk be részletesen.

\subsubsection{Email cím megadása}

Az azonosítási folyamat elindításához a felhasználónak először meg kell adnia az email címét a CryptID.email weboldalán.

\subsubsection{Megerősítő email kiküldése}

A CryptID.email szerveroldali komponense ezután generál egy nyolc karakter hosszúságú, úgynevezett email tokent, melyet SendGriden\footnote{\url{https://sendgrid.com/}} keresztül kiküld a felhasználó által megadott email címre. A token csupán tíz percig érvényes, ezt követően nem alkalmas az email cím igazolására.

\subsubsection{Email token megadása}

Ha a megadott email cím valóban a felhasználó tulajdonában áll, akkor megkapja az említett email tokent is, melyet aztán megadhat a CryptID.email weboldalán. Ezzel a felhasználó igazolta az identitását, elindulhat a privát kulcs kinyerése.

\subsubsection{Privát kulcs generálás}

Az identitás ellenőrzését követően a szerver meghatározza a fájl titkosítása során használt publikus paramétereket, valamint a privát kulcs generálásához szükséges mesterkulcsot a titkos fájlban tárolt azonosítón keresztül.  Ezeket a korábban megadott email cím mellett továbbítja a PKG felé, mely a \texttt{CryptID.extract} függvényt meghívva előállítja a publikus kulcs privát párját. Végül a szerver a privát kulcsot elküldi a kliensnek.

\subsubsection{A szimmetrikus kulcs visszafejtése}

A privát kulcs birtokában a kliens megpróbálkozhat a titkos fájlban tárolt szimmetrikus kulcs visszafejtésével. Ehhez a kliens a CrpytID könyvtár \texttt{decrypt} függvényét használja. Amennyiben a privát kulcs helyes, azaz valóban a szóban forgó felhasználó email címével titkosították a szimmetrikus kulcsot, akkor a visszafejtés sikeres lesz. Ellenkező esetben a folyamat itt megáll, a felhasználó nem fér hozzá az eredeti tartalomhoz.

\subsubsection{Az eredeti fájl visszaállítása}

A folyamat utolsó lépése az eredeti fájl tartalmának visszafejtése, mely az előzőleg nyert szimmetrikus kulcs és a WebCrypto API segítségével történik. A memóriában előálló nyílt reprezentációt ezután a felhasználó a háttértárra mentheti. A visszafejtési folyamat ezzel véget ért, visszaállítottuk az eredeti fájlt.

\subsection{Összegzés}

A CryptID.email különböző eszközökön és platformokon átívelő egységes titkosítási megoldást nyújt. A használatához nem szükséges sem előzetes kulcscsere, sem a PKI (\textit{public key infrastructure}) használata, hiszen a továbbítandó fájlok titkosításához csupán a címzett email címét kell megadnunk.


\section{Esettanulmány – Személyre szabott zárthelyi}
\label{Section::Applications::Zarthelyi}

Míg az előző alkalmazás a CryptID hordozható és könnyen használható voltát demonstrálta, addig a személyre szabott zárthelyit leíró rövid esettanulmány elsődleges célja a metaadatok beágyazásának bemutatása. Tapasztalataink szerint a metaadatok leginkább egy adott szakterületen belül lehetnek hasznosak, ezért kézenfekvő választás volt az esettanulmány témájának az általunk jól ismert zárthelyi dolgozatok világa.

Az alkalmazás hallgatónként egyedi elektronikus feladatsorok összeállítását teszi lehetővé, melyek a titkosításnak köszönhetően akár napokkal a dolgozat időpontja előtt elhelyezhetők az egyetemi számítógépeken.

\subsection{Motiváció}

A motiváció a csalások megelőzésében gyökerezik: úgy gondoljuk, hogy a hálózati hozzáférés korlátozásával szemben sokkal hatásosabb eszközt jelent a visszaélések megakadályozásában, ha minden hallgató eltérő feladatsorral rendelkezik. Ennek kezelése azonban a különböző feladatsorok előállításán és ellenőrzésén felül további problémákat is felvet. Ahhoz, hogy ez az elképzelés valóban csökkentse a csalások számát, biztosítani kell, hogy minden hallgató csak a saját feladatsorához férjen hozzá és csak a zárthelyi időpontjában. 

Úgy gondoltuk, hogy erre a célra a CryptID biztosította IBE egy valódi megoldást nyújthat.

\subsection{Megvalósítás}

A rendszer elméleti megvalósítását a publikus kulcs megtervezésén keresztül mutatjuk be. A következőkben egyenként hozzáadunk három mezőt a publikus kulcshoz, kifejtve egyúttal az adott mező implementációval való kapcsolatát. A publikus kulcs tartalmának ilyen módon történő leírása teret enged az alkalmazás összes fontos aspektusának bemutatására.

\subsubsection{Neptun-kód}

A publikus kulcs nélkülözhetetlen összetevője egy, a hallgatók azonosítására szolgáló mező. Természetes választás a Neptun-kód, mellyel Karunk minden hallgatója rendelkezik. Előnyös tulajdonsága továbbá, hogy ellenőrzése már kiépített infrastruktúrán keresztül megtehető: a privát kulcs generálása előtt a hallgató a hálózati azonosítóját használva bejelentkezik, ezzel igazolva az identitását.

Önmagában mindazonáltal a Neptun-kód nem elégséges. Ha a feladatsort csak a címzett hallgató azonosítójával titkosítjuk, akkor a visszafejtéshez használt privát kulcs tetszőleges, a hallgatónak szánt feladatsor visszafejtésére alkalmas lesz!

\subsubsection{Dátum}

Érdemes tehát az előbb említett probléma megakadályozására egy újabb mezővel, a dátummal kiegészíteni a publikus kulcsot. Titkosításkor a Neptun-kód mellett a zárthelyi dolgozat dátumát is felhasználjuk, a privát kulcs generálása előtt pedig a rendszer automatikusan elhelyezi az aktuális dátumot a publikus kulcsban. Ez rendkívül egyszerűen megtehető, hiszen csak egy JSON objektumot kell egy újabb mezővel kiegészíteni.

Gondot jelenthet ugyanakkor, amennyiben a hallgató egy napon több zárthelyit is ír. Erre egy lehetséges válasz a publikus kulcsban elhelyezett dátum finomságának növelése, például \textit{év-hónap-nap} helyett \textit{év-hónap-nap-óra} mezők használatával. Jobb megoldást kínál azonban a publikus kulcs további bővítése.

\subsubsection{Tárgy}

A bővítés ötlete azon alapszik, hogy a titkosításhoz felhasználhatjuk a szóban forgó tárgy nevét is. Nem szabad azonban elfelejteni, hogy csak olyan mezőt érdemes elhelyeznünk a publikus kulcsban, melyet aztán a privát kulcs generálásakor ellenőrizni tudunk.

A \textit{tárgy} mezőt ezért a dátumhoz hasonlóan közvetlenül a szerveroldalon helyezzük el a publikus kulcsban, a privát kulcs generálása előtt, megkerülve ezzel az ellenőrzés szükségességét. Ekkor azonban biztosítani kell, hogy a mező értéke megbízható forrásból származzék. Ehhez a hallgató által a zárthelyi megírásához használt számítógép IP címe nyújthat alapot. Ha az IP cím alapján meghatározható a terem, ahol az eszköz elhelyezkedik, akkor ezt az információt a teremfoglalási adatbázissal összekapcsolva egyértelműen meg tudjuk mondani, hogy a hallgató épp milyen tárgyhoz kötődő zárthelyin vesz részt.

Ez a mező tehát a fizikai hely közvetett ellenőrzésének lehetőségét nyújtja.

\subsection{Összegzés}

A vázolt alkalmazás garantálja, hogy az egyes feladatsorok visszafejtése csak a kívánt hallgató által, a kívánt időben és helyen végezhető el. Mindezen követelmények kikényszerítése automatikusan, a publikus kulcsban elhelyezett metaadatok segítségével történik.

Ennek köszönhetően a dolgozatokhoz használt feladatsorok jóval a zárthelyi tényleges időpontja előtt elhelyezhetők a számítógépeken, valamint akár hálózati korlátozás is alkalmazható – csupán a PKG-hez történő hozzáférést kell biztosítani.

Mindamellett nem szabad elfelejtkeznünk arról, hogy CryptID alternatívája lehet az egyszerű szimmetrikus titkosítás alkalmazása: a feladatsorok előállításával egy időben minden feladatsorhoz generálunk egy szimmetrikus kulcsot, melyhez a hallgató csak sikeres azonosítás után férhet hozzá. Csakhogy az IBE egyik lehetséges alkalmazási területét épp az ilyen, sok szimmetrikus kulcs menedzselését igénylő rendszerek kiváltása jelenti, hiszen a mesterkulcson felül semmilyen más kulcs tárolására nincsen szükség.
