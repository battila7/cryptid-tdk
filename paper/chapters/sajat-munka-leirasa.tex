\chapter*{Saját munka leírása}
\addcontentsline{toc}{chapter}{Saját munka leírása}

\if\printName1
    \paragraph{Bagossy Attila}
\fi

\begin{itemize}
    \item
    Implementáltam a CryptID.ref Tate párosítását, valamint kialakítottam a publikus API-t.

    \item
    Elkészítettem a CryptID.wasm és a CryptID.js közötti interoperabilitási réteget, illetve implementáltam a CryptID.js-t.

    \item
    Teljesítményméréseket készítettem, melyek mind asztali, mind mobil eszközön demonstrálják a könyvtárat alkotó legfontosabb függvények futási idejét.

    \item
    Megvalósítottam a CryptID.email alkalmazást, mely egy gyakorlati példán keresztül mutatja be a CryptID használatát.

\end{itemize}

\if\printName1
    \paragraph{Vécsi Ádám}
\else
    \hfill\break
    \hfill\break
    \hfill\break
\fi

\begin{itemize}
    \item
    A CryptID.ref könyvtárban elkészítettem a következőket:

    \begin{itemize}
        \item
        komplex számokon értelmezett műveletek,

        \item
        elliptikus görbe aritmetika,

        \item
        a Boneh-Franklin IBE függvényei.
    \end{itemize}

    \item
    A referencia-implementációt alapul véve a fentieket, illetve a Tate párosítást implementáltam a CryptID.wasm részeként is, integrálva a GMP és OpenSSL könyvtárakat.

    \item
    Kidolgoztam a személyre a szabott zárthelyi alkalmazás ötletét.

\end{itemize}
